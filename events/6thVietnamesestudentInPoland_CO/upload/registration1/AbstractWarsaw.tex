
\documentclass[prl,aps,superscriptaddress]{revtex4}
%%%%%%%%%%%%%%%%%%%%%%%%%%%%%%%%%%%%%%%%%%%%%%%%%%%%%%%%%%%%%%%%%%%%%%%%%%%%%%%%%%%%%%%%%%%%%%%%%%%%%%%%%%%%%%%%%%%%%%%%%%%%%%%%%%%%%%%%%%%%%%%%%%%%%%%%%%%%%%%%%%%%%%%%%%%%%%%%%%%%%%%%%%%%%%%%%%%%%%%%%%%%%%%%%%%%%%%%%%%%%%%%%%%%%%%%%%%%%%%%%%%%%%%%%%%%
%TCIDATA{OutputFilter=Latex.dll}
%TCIDATA{Version=5.00.0.2552}
%TCIDATA{<META NAME="SaveForMode" CONTENT="1">}
%TCIDATA{LastRevised=Monday, June 10, 2013 17:46:08}
%TCIDATA{<META NAME="GraphicsSave" CONTENT="32">}

\input{tcilatex}

\begin{document}

\title{Stationary Josephson current between $d$-wave superconductors with
charge density waves: angular dependences and violations of the
corresponding-states relationship}
\author{\underline{Alexander M. Gabovich}}
\affiliation{Institute of Physics, NASU, 46, Nauka Ave., Kyiv 03028, Ukraine}
\author{Mai Suan Li}
\affiliation{Institute of Physics, PAN, 32/46, Al. Lotnik\'{o}w, PL-02-668 Warsaw, Poland}
\author{Henryk Szymczak}
\affiliation{Institute of Physics, PAN, 32/46, Al. Lotnik\'{o}w, PL-02-668 Warsaw, Poland}
\author{Alexander I. Voitenko}
\affiliation{Institute of Physics, NASU, 46, Nauka Ave., Kyiv 03028, Ukraine}
\maketitle

Stationary Josephson tunnel current $I_{c}$\ was calculated for junctions
made of superconductors partially gapped by biaxial or unidirectional charge
density waves (CDWs) and possessing a superconducting order parameter of $d$%
-wave symmetry. Specific calculations were carried out for symmetric
junctions between two identical CDW superconducting electrodes and
nonsymmetric ones composed of a CDW superconductor and a conventional
isotropic $s$-wave superconductor. The directionality of tunneling was made
allowance for. In all studied cases, the dependences of $I_{c}$ on the angle 
$\gamma $\ between the chosen crystal direction and the normal to the
junction plane were found to be significantly influenced by CDWs. It was
shown, in particular, that the $d$-wave driven periodicity of $I_{c}(\gamma
) $ in the CDW-free case is transformed into double-period beatings
depending on the parameters of the system. The results of calculation
testify that the orientation-dependent patterns $I_{c}(\gamma )$ measured
for CDW superconductors allow the CDW configuration (unidirectional or
checkerboard) and the symmetry of superconducting order parameter to be
determined.

It was shown that when CDWs are absent or weak, there exists an approximate
proportionality between $I_{c}(x)$ and the product of superconducting energy
gaps $\Delta (x)$ and $\Delta (x^{\prime })$ to the left and to the right of
the tunnel barrier. Here, $x$ is either the reduced temperature, $T/T_{c}$,
where $T_{c}$ is the critical temperature of the superconducting transition,
or one of the parameters characterizing the combined CDW superconducting
phase. However, provided a high directionality of tunneling, CDWs may
violate the law of corresponding states\textrm{.} The proposed method is an
additional one to detect CDWs in cuprates along with the measurements of $%
I_{c}(\gamma )$\ dependences.

\end{document}
